% ==============================================================================
% APRESENTAÇÃO - SISTEMA SOMOS DARUA
% Disciplina: Banco de Dados
% ==============================================================================

\documentclass[aspectratio=169]{beamer}
\usepackage[utf8]{inputenc}
\usepackage[portuguese]{babel}
\usepackage{graphicx}
\usepackage{tikz}
\usepackage{booktabs}
\usepackage{multirow}
\usepackage{xcolor}
\usepackage{listings}

% Tema
\usetheme{Madrid}
\usecolortheme{default}

% Informações do documento
\title{Sistema Somos DaRua}
\subtitle{Projeto de Banco de Dados para Gestão de Doações}
\author{Equipe Somos DaRua}
\institute{Disciplina: Banco de Dados}
\date{\today}

\begin{document}

% ==============================================================================
% SLIDE 1: TÍTULO
% ==============================================================================
\begin{frame}
\titlepage
\end{frame}

% ==============================================================================
% SLIDE 2: SUMÁRIO
% ==============================================================================
\begin{frame}{Sumário}
\tableofcontents
\end{frame}

% ==============================================================================
% SEÇÃO 1: INTRODUÇÃO
% ==============================================================================
\section{Introdução}

\begin{frame}{Visão Geral do Projeto}
\begin{block}{Contexto}
Sistema de gestão de doações para organizações sociais que atendem pessoas em situação de vulnerabilidade.
\end{block}

\vspace{0.5cm}

\begin{columns}[T]
\column{0.5\textwidth}
\textbf{Objetivo Principal}
\begin{itemize}
    \item Gerenciar doadores e doações
    \item Controlar distribuição aos beneficiários
    \item Organizar campanhas de arrecadação
    \item Gerar relatórios e estatísticas
\end{itemize}

\column{0.5\textwidth}
\textbf{Tecnologias}
\begin{itemize}
    \item \textbf{SGBD:} MySQL 8.0
    \item \textbf{Backend:} Python 3.10+
    \item \textbf{Frontend:} Streamlit
    \item \textbf{Charset:} UTF-8 (utf8mb4)
\end{itemize}
\end{columns}
\end{frame}

\begin{frame}{Problema e Solução}
\begin{block}{Problema}
Organizações sociais enfrentam dificuldades para:
\begin{itemize}
    \item Rastrear histórico de doações
    \item Coordenar logística de entrega
    \item Gerenciar múltiplas campanhas simultaneamente
    \item Associar itens doados às necessidades específicas
\end{itemize}
\end{block}

\pause

\begin{alertblock}{Solução Proposta}
Sistema centralizado com banco de dados relacional que permite:
\begin{itemize}
    \item Cadastro completo de todas as entidades
    \item Sistema de duas fases (recebimento $\rightarrow$ distribuição)
    \item Relacionamentos N:N para flexibilidade
    \item Integridade referencial garantida
\end{itemize}
\end{alertblock}
\end{frame}

% ==============================================================================
% SEÇÃO 2: MODELAGEM CONCEITUAL
% ==============================================================================
\section{Modelagem Conceitual}

\begin{frame}{Entidades Principais}
\begin{center}
\begin{tikzpicture}[
    entity/.style={rectangle, draw, fill=primarycolor!20, text width=3cm, text centered, minimum height=1cm},
    relationship/.style={diamond, draw, fill=secondarycolor!20, text width=2cm, text centered, minimum height=0.8cm, aspect=2}
]

% Entidades centrais
\node[entity] (doador) at (0,3) {\textbf{DOADOR}};
\node[entity] (doacao) at (4,3) {\textbf{DOAÇÃO}};
\node[entity] (beneficiario) at (8,3) {\textbf{BENEFICIÁRIO}};

% Entidades secundárias
\node[entity] (campanha) at (4,0) {\textbf{CAMPANHA}};
\node[entity] (objeto) at (8,0) {\textbf{OBJETO DOÁVEL}};
\node[entity] (voluntario) at (0,0) {\textbf{VOLUNTÁRIO}};

% Relações sem setas (somente traço)
\draw[-, thick] (doador) -- node[above] {faz} (doacao);
\draw[-, thick] (doacao) -- node[above] {beneficia} (beneficiario);
\draw[-, thick] (campanha) -- node[right] {organiza} (doacao);
\draw[-, thick] (doacao) -- node[right] {contém} (objeto);
\draw[-, thick] (voluntario) -- node[below] {distribui} (doacao);

\end{tikzpicture}
\end{center}

\vspace{0.3cm}
\textbf{Total:} 8 entidades principais + 5 relacionamentos N:N = \textbf{13 tabelas}
\end{frame}


\begin{frame}{Relacionamentos e Cardinalidades}
\begin{table}
\centering
\small
\begin{tabular}{lccl}
\toprule
\textbf{Relacionamento} & \textbf{Tipo} & \textbf{Tabela} & \textbf{Descrição} \\
\midrule
Doador $\rightarrow$ Doação & 1:N & - & Um doador faz várias doações \\
Campanha $\rightarrow$ Doação & 1:N & - & Uma campanha tem várias doações \\
\midrule
Doação $\leftrightarrow$ Beneficiário & N:M & \texttt{Recebe} & Doação para vários beneficiários \\
Doação $\leftrightarrow$ Objeto & N:M & \texttt{Contem} & Doação contém vários objetos \\
Doação $\leftrightarrow$ Voluntário & N:M & \texttt{Possui} & Voluntários distribuem doações \\
Campanha $\leftrightarrow$ Necessidade & N:M & \texttt{Promove} & Campanha promove necessidades \\
Objeto $\leftrightarrow$ Campanha & N:M & \texttt{Associa} & Objetos associados a campanhas \\
\bottomrule
\end{tabular}
\end{table}

\vspace{0.3cm}
\begin{alertblock}{Destaque}
5 tabelas de relacionamento garantem flexibilidade e evitam redundância.
\end{alertblock}
\end{frame}

% ==============================================================================
% SEÇÃO 3: MODELO LÓGICO
% ==============================================================================
\section{Modelo Lógico}

\begin{frame}{Esquema Relacional - Tabelas Principais (1/2)}
\scriptsize
\begin{columns}[T]
\column{0.5\textwidth}
\textbf{Doador}
\begin{itemize}
    \item \textcolor{blue}{idDoador} (PK)
    \item Nome (NOT NULL)
    \item Telefone, Email
    \item Endereço completo (7 campos)
\end{itemize}

\vspace{0.3cm}
\textbf{Beneficiario}
\begin{itemize}
    \item \textcolor{blue}{idBeneficiario} (PK)
    \item Nome (NOT NULL)
    \item Idade, Genero
    \item Descricao
\end{itemize}

\vspace{0.3cm}
\textbf{Doacao} (Central)
\begin{itemize}
    \item \textcolor{blue}{idDoacao} (PK)
    \item DataCriacao (DEFAULT NOW)
    \item DataEntrega
    \item \textcolor{red}{Doador\_idDoador} (FK, NOT NULL)
    \item \textcolor{red}{Campanha\_idCampanha} (FK)
\end{itemize}

\column{0.5\textwidth}
\textbf{CampanhaDoacao}
\begin{itemize}
    \item \textcolor{blue}{idCampanha} (PK)
    \item Nome (NOT NULL)
    \item DataInicio, DataTermino
    \item Descricao
\end{itemize}

\vspace{0.3cm}
\textbf{ObjetoDoavel}
\begin{itemize}
    \item \textcolor{blue}{idObjeto} (PK)
    \item Nome (NOT NULL)
    \item Descricao, Categoria
    \item \textcolor{red}{PontoColeta\_id} (FK)
\end{itemize}

\vspace{0.3cm}
\textbf{PontoColeta}
\begin{itemize}
    \item \textcolor{blue}{idPontoColeta} (PK)
    \item Responsavel (NOT NULL)
    \item Endereço completo
\end{itemize}

\end{columns}
\end{frame}

\begin{frame}{Esquema Relacional - Tabelas N:N}
\begin{table}
\centering
\small
\begin{tabular}{lll}
\toprule
\textbf{Tabela} & \textbf{Chave Composta} & \textbf{Função} \\
\midrule
\texttt{Contem} & (Doacao\_id, Objeto\_id) & Itens da doação \\
\texttt{Recebe} & (Beneficiario\_id, Doacao\_id) & Quem recebeu \\
\texttt{Possui} & (Doacao\_id, Voluntario\_id) & Quem distribuiu \\
\texttt{Promove} & (Campanha\_id, Necessidade\_id) & Necessidades da campanha \\
\texttt{Associa} & (Objeto\_id, Campanha\_id) & Objetos da campanha \\
\bottomrule
\end{tabular}
\end{table}

\vspace{0.5cm}

\begin{block}{Normalização}
\begin{itemize}
    \item \textbf{3FN (Terceira Forma Normal):} Todas as tabelas estão normalizadas
    \item \textbf{Sem redundância:} Informações não duplicadas
    \item \textbf{Integridade:} Foreign keys com CASCADE/RESTRICT apropriados
\end{itemize}
\end{block}
\end{frame}

% ==============================================================================
% SEÇÃO 4: IMPLEMENTAÇÃO FÍSICA
% ==============================================================================
\section{Implementação Física}

\begin{frame}{Especificações Técnicas}
\begin{block}{Configuração do Banco}
\begin{itemize}
    \item \textbf{SGBD:} MySQL 8.0+
    \item \textbf{Engine:} InnoDB (suporte a transações ACID)
    \item \textbf{Charset:} utf8mb4 (suporte completo a Unicode)
    \item \textbf{Collation:} utf8mb4\_unicode\_ci
\end{itemize}
\end{block}

\vspace{0.5cm}

\begin{columns}[T]
\column{0.5\textwidth}
\textbf{Vantagens InnoDB}
\begin{itemize}
    \item Transações ACID
    \item Foreign keys nativas
    \item Row-level locking
    \item Crash recovery
\end{itemize}

\column{0.5\textwidth}
\textbf{Decisões de Design}
\begin{itemize}
    \item AUTO\_INCREMENT em PKs
    \item Índices em FKs
    \item DEFAULT values estratégicos
    \item CHECK constraints
\end{itemize}
\end{columns}
\end{frame}

\begin{frame}{Foreign Keys e Integridade Referencial}
\begin{table}
\centering
\scriptsize
\begin{tabular}{llll}
\toprule
\textbf{FK} & \textbf{Referencia} & \textbf{ON DELETE} & \textbf{Razão} \\
\midrule
Doacao.Doador\_id & Doador.id & RESTRICT & Preserva histórico \\
Doacao.Campanha\_id & Campanha.id & SET NULL & Doação permanece \\
Objeto.PontoColeta\_id & PontoColeta.id & SET NULL & Objeto fica sem ponto \\
\midrule
Contem.Doacao\_id & Doacao.id & CASCADE & Remove itens junto \\
Recebe.Doacao\_id & Doacao.id & CASCADE & Remove vínculos \\
Possui.Doacao\_id & Doacao.id & CASCADE & Remove vínculos \\
\bottomrule
\end{tabular}
\end{table}

\vspace{0.3cm}

\begin{alertblock}{Estratégia}
\begin{itemize}
    \item \textbf{RESTRICT:} Protege dados importantes (doadores com doações)
    \item \textbf{SET NULL:} Mantém registro mas remove vínculo opcional
    \item \textbf{CASCADE:} Remove dados dependentes automaticamente
\end{itemize}
\end{alertblock}
\end{frame}

\begin{frame}{Índices e Otimização}
\begin{columns}[T]
\column{0.5\textwidth}
\textbf{Índices Primários}
\begin{itemize}
    \item PK automático (clustered)
    \item Todos os IDs indexados
    \item Busca $O(\log n)$
\end{itemize}

\vspace{0.3cm}
\textbf{Índices de Foreign Keys}
\begin{itemize}
    \item \texttt{idx\_doador} em Doacao
    \item \texttt{idx\_campanha} em Doacao
    \item \texttt{idx\_ponto} em Objeto
\end{itemize}

\column{0.5\textwidth}
\textbf{Índices de Busca}
\begin{itemize}
    \item \texttt{idx\_nome} em Doador
    \item \texttt{idx\_nome} em Beneficiario
    \item \texttt{idx\_categoria} em Objeto
    \item \texttt{idx\_cidade} em PontoColeta
\end{itemize}

\vspace{0.3cm}
\textbf{Índice Composto}
\begin{itemize}
    \item \texttt{idx\_data\_campanha}
    \item (DataInicio, DataTermino)
    \item Otimiza consultas por período
\end{itemize}
\end{columns}

\vspace{0.5cm}
\begin{block}{Performance}
Total de \textbf{15 índices} estrategicamente posicionados para otimizar queries mais frequentes.
\end{block}
\end{frame}

\begin{frame}{Constraints}

\begin{block}{CHECK Constraints}
\small
\begin{itemize}
    \item \textbf{Beneficiário.Idade:} deve ser $\geq 0$
    \item \textbf{CampanhaDoacao:} DataTermino $\geq$ DataInicio
\end{itemize}
\end{block}

\vspace{0.2cm}

\begin{block}{NOT NULL Constraints}
\small
Campos obrigatórios:
\begin{itemize}
    \item Nomes (Doador, Beneficiário, Voluntário, etc.)
    \item Doacao.DataCriacao, Doacao.Doador\_idDoador
    \item PontoColeta.Responsavel
\end{itemize}
\end{block}

\end{frame}

\begin{frame}{Validações da Aplicação}

\begin{alertblock}{Validações em Python (Model Layer)}
\small
\begin{itemize}
    \item Formato de email válido (contém @)
    \item CEP com 8 dígitos
    \item Estado com exatamente 2 caracteres (UF)
\end{itemize}
\end{alertblock}

\end{frame}


% ==============================================================================
% SEÇÃO 5: QUERIES E OPERAÇÕES
% ==============================================================================
\section{Queries e Operações}

\begin{frame}{Consultas Estratégicas - Joins Complexos}
\textbf{1. Doações com Informações Completas}
\begin{itemize}
    \item JOIN entre Doacao, Doador e Campanha
    \item Retorna histórico completo de cada doação
    \item Inclui nome do doador e campanha associada
\end{itemize}

\vspace{0.3cm}

\textbf{2. Objetos de uma Doação}
\begin{itemize}
    \item JOIN com tabela N:N \texttt{Contem}
    \item Lista todos os itens doados
    \item Agrupa por categoria
\end{itemize}

\vspace{0.3cm}

\textbf{3. Beneficiários de uma Doação}
\begin{itemize}
    \item JOIN com tabela N:N \texttt{Recebe}
    \item Identifica quem recebeu cada doação
    \item Perfil demográfico (idade, gênero)
\end{itemize}
\end{frame}

\begin{frame}{Consultas Estratégicas - Agregações}
\textbf{4. Top 10 Doadores}
\begin{itemize}
    \item GROUP BY em Doador
    \item COUNT de doações
    \item ORDER BY para ranking
\end{itemize}

\vspace{0.3cm}

\textbf{5. Doações por Período}
\begin{itemize}
    \item DATE\_FORMAT para agrupar por mês
    \item WHERE com BETWEEN para filtrar datas
    \item Útil para gráficos de evolução temporal
\end{itemize}

\vspace{0.3cm}

\textbf{6. Campanhas Ativas}
\begin{itemize}
    \item WHERE CURRENT\_DATE BETWEEN DataInicio AND DataTermino
    \item Filtra campanhas em andamento
    \item Calcula dias restantes
\end{itemize}
\end{frame}

\begin{frame}{Sistema de Duas Fases}
\begin{block}{Fluxo de Estados da Doação}
\begin{center}
\begin{tikzpicture}[
    state/.style={rectangle, rounded corners, draw, fill=blue!20, text width=3cm, text centered, minimum height=1cm},
    arrow/.style={->, thick}
]

\node[state] (recebida) at (0,0) {\textbf{RECEBIDA}\\{\small Doador entrega\\no Ponto de Coleta}};
\node[state] (distribuida) at (5,0) {\textbf{DISTRIBUÍDA}\\{\small Entregue aos\\Beneficiários}};

\draw[arrow] (recebida) -- node[above] {Distribuir()} (distribuida);

\end{tikzpicture}
\end{center}
\end{block}

\vspace{0.5cm}

\textbf{Operações de Banco de Dados:}
\begin{enumerate}
    \item \textbf{Fase 1:} INSERT em Doacao + INSERTs em Contem
    \item \textbf{Fase 2:} INSERTs em Recebe + INSERTs em Possui + UPDATE Doacao.DataEntrega
\end{enumerate}

\vspace{0.3cm}
\begin{alertblock}{Transações}
Cada fase usa BEGIN/COMMIT para garantir consistência (ACID).
\end{alertblock}
\end{frame}

% ==============================================================================
% SEÇÃO 6: FUNCIONALIDADES AVANÇADAS
% ==============================================================================
\section{Funcionalidades Avançadas}

\begin{frame}{Dashboard e Métricas em Tempo Real}
\begin{columns}[T]
\column{0.6\textwidth}
\textbf{Consultas Principais:}
\begin{enumerate}
    \item Total de doadores cadastrados
    \item Total de beneficiários
    \item Total de doações (com status)
    \item Doações por categoria
    \item Evolução mensal (6 meses)
    \item Top 10 doadores
\end{enumerate}

\column{0.4\textwidth}
\textbf{Performance:}
\begin{itemize}
    \item Índices otimizados
    \item Queries agregadas
    \item Cache na aplicação
    \item Tempo médio: \textless 100ms
\end{itemize}
\end{columns}

\vspace{0.5cm}

\begin{block}{Visualizações}
Dados do banco alimentam gráficos interativos:
\begin{itemize}
    \item Gráfico de pizza (doações por categoria)
    \item Gráfico de barras (top doadores)
    \item Gráfico de linha (evolução temporal)
\end{itemize}
\end{block}
\end{frame}

\begin{frame}{Buscas e Filtros Avançados}
\textbf{1. Busca por Localização}
\begin{itemize}
    \item WHERE com LIKE em Cidade/Estado
    \item Índice em Cidade para performance
    \item Útil para logística de entrega
\end{itemize}

\vspace{0.3cm}

\textbf{2. Busca por Perfil Demográfico}
\begin{itemize}
    \item Filtros combinados: Genero + faixa etária
    \item WHERE com múltiplas condições
    \item Identifica público-alvo de campanhas
\end{itemize}

\vspace{0.3cm}

\textbf{3. Busca por Período e Status}
\begin{itemize}
    \item WHERE DataCriacao BETWEEN + status
    \item Relatórios mensais/anuais
    \item Análise de tendências
\end{itemize}
\end{frame}

\begin{frame}{Migrations e Evolução do Schema}
\begin{block}{Sistema de Migrations}
\begin{itemize}
    \item Versionamento do schema
    \item Scripts SQL incrementais
    \item Rastreabilidade de mudanças
\end{itemize}
\end{block}

\vspace{0.3cm}

\textbf{Exemplos de Migrations Implementadas:}
\begin{enumerate}
    \item \texttt{add\_doacoes\_detalhes.sql}
    \begin{itemize}
        \item Adiciona: TipoDoacao, DescricaoItem, Quantidade, Unidade
        \item ALTER TABLE com ADD COLUMN
    \end{itemize}
    
    \item \texttt{add\_fks\_doacoes.sql}
    \begin{itemize}
        \item Reforça foreign keys
        \item Adiciona ON UPDATE CASCADE
    \end{itemize}
    
    \item \texttt{add\_meta\_campanhas.sql}
    \begin{itemize}
        \item Adiciona sistema de metas
        \item MetaValor, MetaUnidade
    \end{itemize}
\end{enumerate}
\end{frame}

% ==============================================================================
% SEÇÃO 7: BACKUP E MANUTENÇÃO
% ==============================================================================
\section{Backup e Manutenção}

\begin{frame}{Estratégia de Backup}
\begin{block}{Backup Completo Diário}
\begin{itemize}
    \item Comando: \texttt{mysqldump somos\_darua}
    \item Horário: 02:00 (menor tráfego)
    \item Retenção: 30 dias
    \item Armazenamento: Local + Nuvem
\end{itemize}
\end{block}

\vspace{0.3cm}

\begin{columns}[T]
\column{0.5\textwidth}
\textbf{Tipos de Backup:}
\begin{enumerate}
    \item Schema only (estrutura)
    \item Data only (dados)
    \item Full backup (completo)
\end{enumerate}

\column{0.5\textwidth}
\textbf{Restore:}
\begin{itemize}
    \item Tempo médio: 5-10 min
    \item Testado mensalmente
    \item Documentado
\end{itemize}
\end{columns}

\vspace{0.5cm}

\begin{alertblock}{Automação}
Cron job configurado para backup automático todas as noites.
\end{alertblock}
\end{frame}

\begin{frame}{Manutenção}
\textbf{Rotinas de Manutenção:}
\begin{itemize}
    \item \textbf{OPTIMIZE TABLE:} Semanal (domingo, 03:00)
    \item \textbf{ANALYZE TABLE:} Após grandes inserções
    \item \textbf{CHECK TABLE:} Mensal
\end{itemize}

\vspace{0.5cm}

\textbf{Monitoramento:}
\begin{itemize}
    \item Tamanho das tabelas (crescimento esperado)
    \item Performance de queries (slow query log)
    \item Uso de índices (EXPLAIN)
    \item Conexões ativas
\end{itemize}
\end{frame}

\begin{frame}{Métricas de Monitoramento}
\begin{block}{Métricas Atuais}
\begin{itemize}
    \item Banco: \textasciitilde 15 MB
    \item Queries médias: 50--100 ms
    \item 13 tabelas + 15 índices
    \item Taxa de hit de índice: 95\%+
\end{itemize}
\end{block}

\vspace{0.4cm}

\textbf{Indicadores acompanhados:}
\begin{itemize}
    \item Crescimento por tabela
    \item Evolução da latência de queries
    \item Eficiência dos índices ao longo do tempo
\end{itemize}
\end{frame}


% ==============================================================================
% SEÇÃO 8: RESULTADOS
% ==============================================================================
\section{Resultados}

\begin{frame}{Estatísticas do Sistema}
\begin{columns}[T]
\column{0.5\textwidth}
\textbf{Estrutura do Banco}
\begin{itemize}
    \item 8 tabelas principais
    \item 5 tabelas N:N
    \item 15 índices otimizados
    \item 12 foreign keys
    \item 3 CHECK constraints
\end{itemize}

\vspace{0.5cm}
\textbf{Capacidades}
\begin{itemize}
    \item Suporta milhares de registros
    \item Queries em \textless 100ms
    \item Integridade garantida
    \item Escalável
\end{itemize}

\column{0.5\textwidth}
\textbf{Funcionalidades Implementadas}
\begin{itemize}
    \item CRUD completo (8 entidades)
    \item Sistema de duas fases
    \item Relacionamentos N:N
    \item Buscas avançadas
    \item Agregações complexas
    \item Dashboard em tempo real
\end{itemize}

\vspace{0.5cm}
\textbf{Qualidade}
\begin{itemize}
    \item 3FN (normalizado)
    \item Sem redundância
    \item Documentado
    \item Testado
\end{itemize}
\end{columns}
\end{frame}

\begin{frame}{Decisões de Design - Justificativas}
\begin{table}
\centering
\scriptsize
\begin{tabular}{lll}
\toprule
\textbf{Decisão} & \textbf{Alternativa} & \textbf{Justificativa} \\
\midrule
InnoDB & MyISAM & Suporte a FK e transações \\
N:N com tabelas & Colunas multi-valor & Normalização e flexibilidade \\
DataEntrega opcional & Obrigatória & Fase 1 não tem entrega ainda \\
SET NULL em Campanha & CASCADE & Doações sem campanha são válidas \\
RESTRICT em Doador & CASCADE & Preserva histórico completo \\
utf8mb4 & utf8 & Suporte completo a emojis/Unicode \\
Índices compostos & Múltiplos simples & Otimiza queries de período \\
\bottomrule
\end{tabular}
\end{table}

\vspace{0.5cm}

\begin{alertblock}{Princípio}
Cada decisão de design foi tomada considerando:
\begin{itemize}
    \item Requisitos funcionais
    \item Performance
    \item Integridade de dados
    \item Manutenibilidade
\end{itemize}
\end{alertblock}
\end{frame}

\begin{frame}{Lições Aprendidas}
\begin{block}{Desafios Técnicos}
\begin{itemize}
    \item Modelar sistema de duas fases sem coluna de status explícita
    \item Balancear normalização com performance
    \item Escolher ON DELETE correto para cada FK
    \item Otimizar queries com múltiplos JOINs
\end{itemize}
\end{block}

\vspace{0.3cm}

\begin{block}{Boas Práticas Aplicadas}
\begin{itemize}
    \item Nomenclatura consistente (PascalCase)
    \item Documentação inline (comentários no SQL)
    \item Versionamento do schema (migrations)
    \item Testes de integridade referencial
    \item Scripts de setup e backup
\end{itemize}
\end{block}
\end{frame}

% ==============================================================================
% SEÇÃO 9: CONCLUSÃO
% ==============================================================================
\section{Conclusão}

\begin{frame}{Objetivos Alcançados}
\begin{itemize}
    \item[\checkmark] Modelagem conceitual completa (ER)
    \item[\checkmark] Normalização até 3FN
    \item[\checkmark] Implementação física otimizada
    \item[\checkmark] Integridade referencial garantida
    \item[\checkmark] Índices estratégicos para performance
    \item[\checkmark] Queries complexas com JOINs e agregações
    \item[\checkmark] Sistema de migrations para evolução
    \item[\checkmark] Backup e recovery automatizados
    \item[\checkmark] Documentação técnica completa
    \item[\checkmark] Aplicação funcional integrada
\end{itemize}

\vspace{0.5cm}

\begin{alertblock}{Resultado}
Sistema completo de banco de dados para gestão de doações, pronto para uso em produção.
\end{alertblock}
\end{frame}

\begin{frame}{Trabalhos Futuros}
\textbf{Melhorias no Banco de Dados:}
\begin{itemize}
    \item Implementar stored procedures para operações complexas
    \item Criar triggers para auditoria automática
    \item Adicionar views materializadas para relatórios
    \item Implementar particionamento de tabelas grandes
    \item Criar índices full-text para buscas textuais
\end{itemize}

\vspace{0.5cm}

\textbf{Funcionalidades Adicionais:}
\begin{itemize}
    \item Sistema de notificações (tabela de logs)
    \item Histórico de alterações (audit trail)
    \item Geolocalização de pontos de coleta (coordenadas)
    \item Sistema de ratings para doadores/voluntários
\end{itemize}
\end{frame}

\begin{frame}{Agradecimentos}
\begin{center}
\Large
Obrigado pela atenção!

\vspace{1cm}

\normalsize
\textbf{Sistema Somos DaRua}\\
Gestão de Doações com Banco de Dados Relacional

\vspace{1cm}

\textit{Dúvidas e Discussão}

\vspace{0.5cm}

\footnotesize
Repositório: github.com/giusfds/DaRua\\
Documentação completa disponível no repositório
\end{center}
\end{frame}

% ==============================================================================
% BACKUP SLIDES (EXTRAS)
% ==============================================================================
\appendix

\begin{frame}{Diagrama ER Completo}
\begin{center}
\tiny
\begin{tikzpicture}[
    entity/.style={rectangle, draw, fill=blue!10, minimum width=2cm, minimum height=0.8cm},
    relationship/.style={diamond, draw, fill=orange!20, minimum width=1.5cm, minimum height=0.6cm, aspect=2},
    attribute/.style={ellipse, draw, fill=green!10, minimum width=1cm, minimum height=0.5cm}
]

% Entidades principais
\node[entity] (doador) at (0,4) {Doador};
\node[entity] (doacao) at (4,4) {Doação};
\node[entity] (beneficiario) at (8,4) {Beneficiário};
\node[entity] (campanha) at (4,1) {Campanha};
\node[entity] (objeto) at (8,1) {Objeto};
\node[entity] (voluntario) at (0,1) {Voluntário};
\node[entity] (ponto) at (10,2.5) {Ponto Coleta};
\node[entity] (necessidade) at (2,0) {Necessidade};

% Relacionamentos
\draw[->, thick] (doador) -- node[above] {1:N} (doacao);
\draw[->, thick] (doacao) -- node[above] {N:M} (beneficiario);
\draw[->, thick] (campanha) -- node[right] {1:N} (doacao);
\draw[->, thick] (doacao) -- node[below right] {N:M} (objeto);
\draw[->, thick] (voluntario) -- node[above] {N:M} (doacao);
\draw[->, thick] (ponto) -- node[right] {1:N} (objeto);
\draw[->, thick] (campanha) -- node[above] {N:M} (necessidade);
\draw[->, thick] (objeto) -- node[right] {N:M} (campanha);

\end{tikzpicture}
\end{center}
\end{frame}

\begin{frame}{Exemplo de Query Complexa}
\textbf{Relatório de Doações por Período com Detalhes:}

\vspace{0.3cm}

\scriptsize
\texttt{SELECT}\\
\texttt{\ \ \ \ d.idDoacao,}\\
\texttt{\ \ \ \ d.DataCriacao,}\\
\texttt{\ \ \ \ do.Nome AS Doador,}\\
\texttt{\ \ \ \ c.Nome AS Campanha,}\\
\texttt{\ \ \ \ COUNT(DISTINCT b.idBeneficiario) AS TotalBeneficiarios,}\\
\texttt{\ \ \ \ COUNT(DISTINCT o.idObjetoDoavel) AS TotalObjetos,}\\
\texttt{\ \ \ \ CASE WHEN d.DataEntrega IS NOT NULL}\\
\texttt{\ \ \ \ \ \ \ \ THEN 'Distribuída' ELSE 'Recebida' END AS Status}\\
\texttt{FROM Doacao d}\\
\texttt{INNER JOIN Doador do ON d.Doador\_idDoador = do.idDoador}\\
\texttt{LEFT JOIN CampanhaDoacao c ON d.CampanhaDoacao\_id = c.idCampanha}\\
\texttt{LEFT JOIN Recebe r ON d.idDoacao = r.Doacao\_idDoacao}\\
\texttt{LEFT JOIN Beneficiario b ON r.Beneficiario\_id = b.idBeneficiario}\\
\texttt{LEFT JOIN Contem ct ON d.idDoacao = ct.Doacao\_idDoacao}\\
\texttt{LEFT JOIN ObjetoDoavel o ON ct.ObjetoDoavel\_id = o.idObjetoDoavel}\\
\texttt{WHERE d.DataCriacao BETWEEN '2024-01-01' AND '2024-12-31'}\\
\texttt{GROUP BY d.idDoacao}\\
\texttt{ORDER BY d.DataCriacao DESC;}

\normalsize
\vspace{0.3cm}
\textbf{Utiliza:} 6 JOINs, 2 agregações, 1 CASE, filtro por data
\end{frame}

\end{document}
